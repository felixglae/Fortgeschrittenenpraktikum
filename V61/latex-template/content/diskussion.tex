\section{Diskussion}
\label{sec:Diskussion}
Die in Abschnitt (\ref{sec:kernspin}) berechneten Spins ergaben sich zu:
\begin{align*}
  I_\mathrm{87} &= 1.69\pm0.10 \\
  I_\mathrm{85} &= 2.65\pm0.23
\end{align*}
Im Vergleich dazu sind die Literaturwerte \cite{Anleitung3} gegeben durch:
\begin{align*}
  I_\mathrm{87} &= 1.5 \\
  I_\mathrm{85} &= 2.5
\end{align*}
Dies entspricht einer Abweichung für $^{87}\mathrm{Rb}$ von ungefähr $12.7\%$ und für $^{85}\mathrm{Rb}$ von ungefähr $6\%$. Dieses Ergebnis bestätigt die dem Experiment zu Grunde liegende
Theorie relativ gut. Die Abweichungen lassen sich größtenteils dadurch erklären, dass das Ablesen oft nur ungenau möglich war und so einige Messwerte, gerade bei höheren Frequenzen, schwer festzustellen waren.\\
Das Isotopenverhältnis wurde zu $^{87}\mathrm{Rb} \approx 31\%$ und $^{85}\mathrm{Rb} \approx 69\%$ bestimmt. Die Literaturwerte \cite{Anleitung3} sind gegeben durch $^{87}\mathrm{Rb} = 27.83\%$ und $^{85}\mathrm{Rb} = 72.17\%$.
Dies entspricht einer Abweichung für $^{87}\mathrm{Rb}$ von ungefähr $11.3\%$ und für $^{85}\mathrm{Rb}$ von ungefähr $4.3\%$. Damit sind in etwa die selben Abweichungen wie bei den Kernspins vorhanden. In diesem Teil war es jedoch noch
schwieriger genau vom Oszilloskopbild abzulesen, die kleinen Abweichungen sprechen daher für eine gute Messung und ein gelungenes Experiment.\\
Wie bei den verwendeten magnetischen Feldstärken zu erwarten spielt der quadratische Zeeman-Effekt eine eher untergeordnete Rolle. Die Energien für diesen sind drei Größenordnungen kleiner als vom linearen Zeeman-Effekt.
