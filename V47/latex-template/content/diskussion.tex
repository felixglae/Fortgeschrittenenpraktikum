\newpage
\section{Diskussion}
\label{sec:Diskussion}
Zunächst wird die Durchführung des Experimentes diskutiert. Probe und Probengehäuse sollten in etwa die gleiche Temperatur besitzen, die
Einstellung der Heizung von Gehäuse und Probe war jedoch unabhängig voneinander. So musste nach subjektiver Einschätzung immer wieder eine Anpassung an
der Heizung des Gehäuses vorgenommen werden. Da das System jedoch träge auf diese Anpassungen reagiert, war es nicht exakt möglich, die Temperaturen gleichzuhalten.
Ein weitere Möglichkeit die Fehler bei der Durchführung verursacht haben könnte ist, dass im Inneren des Rezipienten kein ideales Vakuum existiert und somit
Wärmeverluste durch Konvektion möglich ist. Zusätzlich konnte die Zieltemperatur von $T=\SI{80}{\kelvin}$ nicht erreicht werden und der Messvorgang wurde bei
$T\approx\SI{105}{\kelvin}$ begonnen. \\

\noindent Berechnet wurden für die Debyetemperatur $\Theta_\mathrm{D}$ von Kupfer die folgenden Werte:
\begin{align*}
  \Theta_\mathrm{D,Messung} &= \SI{279.9(2)}{\kelvin} \\
  \Theta_\mathrm{D,Theorie} &= \SI{332.57}{\kelvin}
\end{align*}
Vergleicht man diese mit dem Literaturwert \cite{Anleitung3} $\Theta_\mathrm{D,Lit}=\SI{345}{\kelvin}$, ergeben sich diese Abweichungen:
\begin{align*}
  \Delta\Theta_\mathrm{D,Messung} &= 18.87 \% \\
  \Delta\Theta_\mathrm{D,Theorie} &= 3.60 \%
\end{align*}
Der Wert für die theoretischee Debyetemperatur $\Theta_\mathrm{D,Theorie}$ ist somit wesentlich näher am Literaturwert. Die große Abweichung beim experimentellen Wert $\Theta_\mathrm{D,Messung}$
lässt sich vermutlich auf die oben angeführten Schwierigkeiten bei der Durchführung zurückführen. Besonders das Problem, dass die Messapparatur nicht weit genug abgekühlt werden konnte und somit einige Messwerte im Niedertemperaturbereich fehlen hat sich hier bemerkbar gemacht, da nur die Temperaturen $T<\SI{170}{\kelvin}$ für die Berechnung verwendet wurden. \\

\noindent Als Letztes wird noch die durchschnittlich bestimmte Molwärme mit dem Literaturwert verglichen. Diese ist gegeben durch:
\begin{align*}
  \overline{C}_\mathrm{p}=\SI{23.35(8)}{\joule\mole\per\kelvin}
\end{align*}
Der Literaturwert ist gegeben durch $c_\mathrm{p,Lit}=\SI{0.385}{\kilo\joule\per\kilo\gram\per\kelvin}$ \cite{Anleitung7}. Dieser wird um die Einheit zu korrigieren umgerechnet zu:
\begin{align*}
  c_\mathrm{Lit}\cdot M &= C_\mathrm{p,Lit} \\
  C_\mathrm{p,Lit} &= \SI{24.45}{\joule\mole\per\kelvin}
\end{align*}
Damit ergibt sich eine Abweichung von:
\begin{align*}
  \Delta C_\mathrm{p} = 4.5\%
\end{align*}
Diese Abweichung lässt auf eine relativ gute Bestimmung von der Molwärme von Kupfer schließen. Das Experiment ist also gelungen, auch wenn es große Abweichung bei der experimentell bestimmten
Debyetemperatur gab, sind die anderen erhaltenen Größen relativ genau.
