\input{content/header.tex}

\begin{document}
%1
\maketitle
%2
\begin{frame}{\huge{List of Contents}}
  \begin{columns}
    \column{.5\textwidth}
      \begin{itemize}
        \setlength\itemsep{1em}
        \item About me
        \item Proton Imaging
        \item Motivation for the Bachelor Thesis
        \item Experimental Setup
        \item FE-I4 Module
        \item What's next?
      \end{itemize}
    \column{.5\textwidth}
      \begin{figure}
        \centering
        \includegraphics[width=\textwidth]{figures/tu.jpg}
      \end{figure}
\end{columns}
\end{frame}

%3
\begin{frame}{\huge{About me}}
  \begin{itemize}
    \setlength\itemsep{1em}
    \item Name: Felix Gläsemann
    \item Age: 27
    \item Intended degree: Bachelor in medical physics
    \item Hobbies:
      \begin{itemize}
        \item[\rightarrow] Running, swimming
        \item[\rightarrow] Reading
        \item[\rightarrow] Going out with friends
      \end{itemize}
  \end{itemize}
\end{frame}
%4
\begin{frame}{\huge{Proton Imaging}}
  \begin{columns}
    \column{.6\textwidth}
    \begin{itemize}
      \setlength\itemsep{1em}
      \item Proton beams to create images of tissue
        \begin{itemize}
          \item[\rightarrow] Protons interact with tissue
          \item[\rightarrow] Signals from the interaction get detected
          \item[\rightarrow] Images are reconstructed
        \end{itemize}
        \item Advantages:
        \begin{itemize}
          \item[\rightarrow] Reduced radiation dose
        \end{itemize}
        \item Application in medical imaging and materials science
      \end{itemize}
      \column{.4\textwidth}
        \begin{figure}
          \centering
          \label{fig:rpr}
          \includegraphics[width=.9\textwidth]{figures/pi.png}
          \caption*{Reconstructed proton radiograph (RPR) \cite[p.~99]{pi}}
        \end{figure}
    \end{columns}
\end{frame}
%5
\begin{frame}{\huge{Motivation for the Bachelor Thesis}}
  \begin{columns}[t]
    \column{.6\textwidth}
    \begin{itemize}
      \item Physical and medical application
      \begin{itemize}
        \item[\rightarrow] Improving diagnostics through \textbf{proton imaging} and \textbf{therapy}
        \item[\rightarrow] Better understanding of proton detection for different tunings of the detector
        \item[\rightarrow] Influence of tuning on clustering and measured charge
      \end{itemize}
    \end{itemize}
    \column{.4\textwidth}
  \begin{figure}
    \label{fig:pixde}
    \centering
    \includegraphics[width=.9\textwidth]{figures/detektor.png}
    \caption*{Schematic structure of a pixelated silicon detector \cite[p.~334]{kola}}
  \end{figure}
\end{columns}
\end{frame}
%6
\begin{frame}{\huge{Motivation for the Bachelor Thesis}}
  \begin{itemize}
    \item Comparison between experiment and simulation
  \end{itemize}
\begin{figure}
  \label{fig:landau}
  \centering
  \includegraphics[width=.9\textwidth]{figures/landau_comp.png}
\end{figure}
\end{frame}
%7
\begin{frame}{\huge{Experimental Setup}}
  \begin{figure}
    \centering
    \label{fig:exset}
    \includegraphics[scale=.4]{figures/exset.png}
  \end{figure}
  \begin{itemize}
    \item RW3 phantom used to lower the energy of the protons
    \item FE-I4 used for detection
  \end{itemize}
\end{frame}
%8
\begin{frame}
\frametitle{\huge{FE-I4 Module}}
  \begin{columns}
    \column{.4\textwidth}
    \begin{itemize}
      \item FE-I4
      \begin{itemize}
        \item[\rightarrow] Developed for ATLAS Experiment
        \item[\rightarrow] High-speed data readout
        \item[\rightarrow] Identify and process relevant data from particle collisions
      \end{itemize}
    \end{itemize}
    \column{.6\textwidth}
    \begin{figure}
      \centering
      \label{fig:exset2}
      \includegraphics[scale=.6]{figures/exset2.png}
      \caption*{FE-I4 module \cite[p.~3]{exset2}}
    \end{figure}
  \end{columns}
\end{frame}
%9
\begin{frame}{\huge{What's next?}}
  \begin{itemize}
    \setlength\itemsep{1em}
    \item Building the experimental setup using Allpix Squared
    \item Start the simulation and get first results
    \begin{itemize}
      \item[\rightarrow] Compare results from experiment and simulation
    \end{itemize}
  \end{itemize}
\end{frame}
%9
\begin{frame}
  \centering \Huge\textcolor{tugreen}{\textbf{Questions?}}
\end{frame}

\appendix
\begin{frame}[t]
\frametitle{\huge{Literature}}
\printbibliography{}
\end{frame}
\end{document}
