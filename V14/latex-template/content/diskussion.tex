\newpage
\section{Diskussion}
\label{sec:Diskussion}
\subsection{Absorptionskoeffizienten der einheitlichen Würfel}
Für die Würfel die nur aus einem Material bestehen, sind für Aluminium ein Wert von $\mu=\SI{0.04(1)}{1\per\centi\meter}$ und für Blei ein Wert von $\mu=\SI{0.98(6)}{1\per\centi\meter}$
bestimmt worden. Vergleicht man diese mit den Literaturwerten für die beiden Materialien (siehe Tabelle \ref{tab:tablit}) ergibt sich für Aluminium eine Abweichung von 80,3\% und für
Blei eine Abweichung von 21\%. Es gibt einige Fehlerquellen, die die Messung beeinflussen. Eine ist die manuelle Ausrichtung der Würfel im Strahlengang. Da nur per Auge justiert wurde, ist es schwierig die angegebenen Projektionen genau zu treffen. Gerade bei den Diagonalprojektionen war es schwierig den Würfel optimal zu justieren, da es keine wirklichen Hilfen zur
Orientierung gab. Weiter ist die Strahlungsquelle nicht ideal, dass heißt der Strahl hat in der Realität eine endliche Ausdehnung und ist nicht
punktförmig.\\
\begin{table}
  \centering
  \caption{Literaturwerte für die verschiedenen Materialien \cite{Anleitung3} entnommen für $E=\SI{661,6}{\kilo\electronvolt}$.}
  \label{tab:tablit}
  \begin{tabular}{c c}
    \toprule
		$\mathrm{Material}$ & $\mu_\mathrm{Lit} \: [\si{1\per\centi\meter}]$ \\
    \midrule
    $\mathrm{Delrin}$ & $\num{0,1251}$ \\
		$\mathrm{Blei}$ & $\num{1,2501}$ \\
    $\mathrm{Aluminium}$ & $\num{0,2016}$ \\
    $\mathrm{Messing}$ & $\num{0,6125}$ \\
    $\mathrm{Eisen}$ & $\num{0,5776}$ \\
    \bottomrule
  \end{tabular}
\end{table}
\FloatBarrier
\noindent Die Dichten zur Berechnung der Literaturwerte sind gegeben für Delrin durch $\rho_\mathrm{Del}=\SI{1420}{\kilogram\per\cubic\meter}$ \cite{Anleitung7}, für Blei durch
$\rho_\mathrm{Blei}=\SI{11350}{\kilogram\per\cubic\meter}$, für Eisen durch $\rho_\mathrm{Eisen}=\SI{7860}{\kilogram\per\cubic\meter}$, für den Kupferanteil
des Messing durch $\rho_\mathrm{Kupfer}=\SI{8920}{\kilogram\per\cubic\meter}$ \cite[37]{Anleitung10}, für Aluminium durch $\rho_\mathrm{Alu}=\SI{2700}{\kilogram\per\cubic\meter}$ \cite{Anleitung4} und
für den Zinkanteil des Messing durch $\rho_\mathrm{Zink}=\SI{7140}{\kilogram\per\cubic\meter}$ \cite{Bild2}.
\subsection{Absorptionskoeffizienten des unbekannten Würfel}
\noindent In Tabelle \ref{tab:tabdiv} sind die Abschwächungskoeffizienten aufgelistet die für den unbekannten Würfel bestimmt worden sind. Ebenfalls sind dort schon erste Vermutungen
für das jeweilige Material angestellt worden. Aus der Anleitung ist bekannt, dass Delrin, Blei, Messing, Eisen und Aluminium die möglichen Materialien sein können.
In Tabelle \ref{tab:tablit} sind die Literaturwerte für alle möglichen Materialien aufgelistet. Es ergeben sich die folgenden Abweichungen:
\begin{table}
  \centering
  \caption{Vergleich der Ergebnisse für die Bestimmung der Absorptionskoeffizienten des unbekannten Würfel mit den Literaturwerten.}
  \label{tab:tablit3}
  \begin{tabular}{c c c c}
    \toprule
		$\mathrm{Position ~ des ~ Würfel}$ & $\mu \: [\si{1\per\centi\meter}]$ & $\mathrm{Vermutetes ~ Material}$ & $\mathrm{Abweichung \: [\%]}$  \\
    \midrule
    $\num{1}$ & $\num{0,08(9)}$ & $\mathrm{Delrin}$ & $\num{36}$\\
		$\num{2}$ & $\num{1,04(8)}$ & $\mathrm{Blei}$ & $\num{16,8}$\\
		$\num{3}$ & $\num{0,03(9)}$ & $\mathrm{Delrin}$ & $\num{76}$\\
    $\num{4}$ & $\num{0,04(7)}$ & $\mathrm{Delrin}$ & $\num{68}$\\
    $\num{5}$ & $\num{1,06(8)}$ & $\mathrm{Blei}$ & $\num{15,2}$\\
    $\num{6}$ & $\num{0,07(7)}$ & $\mathrm{Delrin}$ & $\num{44}$\\
    $\num{7}$ & $\num{0,22(9)}$ & $\mathrm{Aluminium}$ & $\num{8,4}$\\
    $\num{8}$ & $\num{0,88(7)}$ & $\mathrm{Blei}$ & $\num{29,6}$\\
    $\num{9}$ & $\num{0,14(9)}$ & $\mathrm{Delrin}$ & $\num{12}$\\
    \bottomrule
  \end{tabular}
\end{table}
\FloatBarrier
\noindent Für die Würfel 3,4 und 6 sind die Werte ebenfalls zu niedrig um einem Material eindeutig zugeordnet zu werden, daher wird dort Delrin vermutet, da dies unter den angegebenen
Materialien den kleinsten Abschwächungskoeffizienten besitzt. Desweiteren sind alle Abweichungen bis auf die von Würfel 7 im zweistelligen Bereich. Daher sind die aufgestellten Vermutungen keinesfalls belegt. \\
Insgesamt lässt sich sagen, dass es durchaus möglich ist mithilfe von Tomographie Aufschluss über die Zusammensetzung gewisser Materialien zu erhalten, allerdings sollten dafür
deutlich mehr Messreihen aufgenommen werden um Fehler wie die ungenaue Datenaufnahme durch das Analyseprogramm zu minimieren. Eine weitere Möglichkeit zur Verbesserung der Messung wäre
eine Optimierung der Justierung der Würfel in Betracht zu ziehen.
