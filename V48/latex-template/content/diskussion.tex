\section{Diskussion}
\label{sec:Diskussion}
Für die beiden Anordnungen ergaben sich die Resonatorlängen zu $L_\mathrm{1}=\SI{126}{\centi\meter}$ bzw. $L_\mathrm{3} = \SI{123}{\centi\meter}$. Diese liegt etwas unter der durch die Stabilitätsbedingung (\ref{eqn:stabi}) gegebenen
maximalen Länge von $L_\mathrm{Theorie}=\SI{140}{\centi\meter}$. Durch minimale Bewegungen an der Apparatur, wurde der Laserbetrieb erheblich gestört, daher ist nicht sicher ob die tatsächlich möglichen Resonatorlängen richtig festgestellt werden konnten. \\
Die Intensitätskurven von sowohl $\mathrm{TEM}_\mathrm{00}$, als auch  $\mathrm{TEM}_\mathrm{01}$ sind relativ genau an der erwarteten Theoriekurve. Auch wenn einige Messdaten erhebliche Abweichungen zur Kurve haben, so befindet sich der Großteil der Messdaten leicht neben oder genau auf der Theoriekurve. Die Abweichungen lassen sich hauptsächlich so erklären, dass es teilweise sehr schwierig war aufgrund der schwankenden Intensität einen genauen Wert für diese festzustellen. \\
Die erwartete $2\pi$ Periodizizät wird wie in Abbildung (\ref{fig:polarisation}) zu sehen sehr genau bestätigt. Die theoretische Wellenlänge $\lambda_\mathrm{Theorie}=\SI{632.8}{\nano\meter}$ liegt innerhalb des Fehlerbereich der gemittelten Wellenlänge $\lambda=\SI{633.845(425)}{\nano\meter}$. Die leichten Abweichungen sind dadurch zu erklären, dass der Abstand $d_\mathrm{n}$ zu den Nebenmaxima nicht ganz genau abgelesen werden konnte.
