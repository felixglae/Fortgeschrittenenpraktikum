\section{Diskussion}
\label{sec:Diskussion}
Die bestimmten Daten für die Aktivierungsenergie $W$ liegen sowohl beim Polarisationsansatz als auch bei der Berechnung über die Stromdichte im Bereich von $\SI{1}{\electronvolt}$. Allerdings ist es auffällig, dass bei
beiden Heizraten die Aktivierungsenergien für die jeweilige Methode eng beieinander liegen, zwischen den beiden Methoden aber Unterschiede um mehrere Standardabweichungen vorliegen. Das deutet auf einen systematischen Fehler bei den verschiedenen Berechnungsarten hin.\\ Ein weiterer Grund für diese Abweichungen ist, dass für die Messung des Untergrundes kein exponentieller sondern ein linearer Fit verwendet worden ist. Wie in den
Abbildungen (\ref{fig:1}) und (\ref{fig:2}) zu sehen, sind die bereinigten Daten für besonders niedrige und besonders hohe Temperaturen eher schlecht bereinigt und es ergibt sich nicht der Verlauf aus Abbildung (\ref{fig:stromdi}).\\
Wie im letzten Abschnitt (\ref{sec:88}) zu sehen liegen die berechneten Relaxationszeiten um sechs bzw. neun Größenordnungen auseinander. Die Ergebnisse für die Relaxationszeiten sind somit nicht verwertbar, da diese hochsensibel von den Aktivierungsenergien abhängig sind und somit sehr schnell Unterschiede von mehreren Größenordnungen auftreten. 
