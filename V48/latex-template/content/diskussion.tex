\section{Diskussion}
\label{sec:Diskussion}
Die bestimmten Daten für die Aktivierungsenergie $W$ liegen sowohl beim Polarisationsansatz als auch bei der Berechnung über die Stromdichte im Bereich von $\SI{1}{\electronvolt}$.
Der Literaturwert für Kaliumbromid ist gegeben durch $E=\SI{0.69}{\electronvolt}$ \cite[5]{Anleitung6}.
Im Vergleich mit den im Versuch bestimmten Daten ergeben sich damit für die erste Heizrate die folgenden Abweichungen:
\begin{align*}
 \mathrm{Abweichung \, Polarisationsansatz}&=3\% \\
 \mathrm{Abweichung \, Stromdichte}&=68\%
\end{align*}
und für die zweite Heizrate:
\begin{align*}
 \mathrm{Abweichung \, Polarisationsansatz}&=19\% \\
 \mathrm{Abweichung \, Stromdichte}&=56\%
\end{align*}
Die Integrationsmethode mithilfe der Stromdichte hat demnach deutlich größere Abweichungen vom Theoriewert als der Polarisationsansatz.
Allerdings ist es auffällig, dass bei
beiden Heizraten die Aktivierungsenergien für die jeweilige Methode eng beieinander liegen, zwischen den beiden Methoden aber Unterschiede um mehrere Standardabweichungen vorliegen. Das deutet auf einen systematischen Fehler bei den verschiedenen Berechnungsarten hin.\\
Wie im letzten Abschnitt (\ref{sec:88}) zu sehen liegen bei den jeweiligen Heizraten die berechneten Relaxationszeiten um acht bzw. zehn Größenordnungen auseinander. Diese hängen hochsensibel von den Aktivierungsenergien ab, wie durch den exponentiellen Zusammenhang in Gleichung (\ref{eqn:fertig}) ersichtlich ist. Somit treten sehr schnell Unterschiede von mehreren Größenordnungen auf.
Der Literaturwert der Relaxationszeit ist gegeben durch $\tau_\mathrm{0}=\SI{1.096 e-13}{\second}$ \cite[5]{Anleitung6}. Die einzige signifikante Abweichung ist für die erste Heizrate:
\begin{align*}
 \mathrm{Abweichung \, Polarisationsansatz}&=9\%
\end{align*}
Die weiteren Relaxationszeiten weichen um mehrere Größenordnungen ab und sind somit nicht aussagekräftig. Für die Polarisationsmethode weicht für die zweite Heizrate die Relaxationszeit um zwei Größenordnungen ab, bei
den Relaxationszeiten die mithilfe der Integrationsmethode bestimmt worden sind ergeben sich Abweichungen von acht Größenordnungen für die erste Heizrate sowie neun Größenordnungen für die zweite Heizrate.
Damit sind die Relaxationszeiten, die mithilfe der Polarisationsmethode berechnet worden sind, wesentlich näher am Literaturwert. Die Polarisationsmethode liefert demnach, wie auch für die Aktivierungsenergie, ein genaueres Ergebnis für die Relaxationszeit.
