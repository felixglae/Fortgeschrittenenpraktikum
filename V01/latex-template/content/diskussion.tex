\newpage
\section{Diskussion}
\label{sec:Diskussion}
\subsection{Wahl der Verzögerungszeit}
In allen drei durchgeführten Messungen zur Bestimmung der Auflösung, lag das Maximum an Counts sowie die Mitte des Plateau im Bereich zwischen $t=\SI{2}{\nano\second}$ und $t=\SI{5}{\nano\second}$, daher scheint unabhänging von der Pulsbreite die ideal einzustellende Verzögerungszeit in diesem Bereich zu liegen.
\subsection{Untergrund}
In diesem Abschnitt sollen kurz die beiden Untergrundraten verglichen werden, diese ergaben sich zu:
\begin{align*}
  U_\mathrm{Fit} &= 3.4\pm0.8 \, \dfrac{\mathrm{Counts}}{\mathrm{Kanal}} \\
  U_\mathrm{Theo}&=0.867\pm0.001 \, \dfrac{\mathrm{Counts}}{\mathrm{Kanal}}
\end{align*}
Die theoretisch mithilfe der Poissonverteilung berechnete Untergrundrate ist wesentlich niedriger als die aus der Ausgleichsrechnung. Damit weicht die theoretisch berechnete um ca 75\%
von der aus dem Fit ab. Allerdings sind beide Werte im Vergleich zur Gesamtanzahl der festgestellten Impulse relativ niedrig und es ist von keiner zu großen Verfälschung der Messung durch den Untergrund auszugehen.
\subsection{Lebensdauer kosmischer Myonen}
Die Lebensdauer der Myonen wurde zu $\tau = \SI{2.06(3)}{\micro\second}$ bestimmt. Der Literaturwert ist gegeben durch $\tau_\mathrm{Lit}=\SI{2.19703(4)}{\micro\second}$ \cite[152]{LitMyo}.
Damit ergibt sich eine Abweichung von 6.23\% zum Literaturwert. Wie schon vorher erwähnt gab es einige Fehlerquellen bei dem Experiment, die so weit wie möglich vermieden werden sollten. Zu sehen ist dies beispielsweise im Vergleich zwischen der realen Suchzeit $t_\mathrm{s,real}=\SI{10.24(1)}{\micro\second}$ und der eingestellt Suchzeit $t_\mathrm{s}=\SI{10}{\micro\second}$.
Die Abweichung kann daher am ehesten durch eine nicht optimale Einstellung der Verzögerungszeit bzw. der gesamten Kalibrierung der Apparatur erklärt werden. Allerdings unterscheidet sich
der Wert für die Lebensdauer der Myonen absolut nur um $\Delta \tau= \SI{0.13703}{\micro\second}$ und ist somit trotzdem relativ gut bestimmt worden.
