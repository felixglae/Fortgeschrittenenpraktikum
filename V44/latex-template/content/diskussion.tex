\section{Diskussion}
\label{sec:Diskussion}
Die durchgeführten Scans zur Justierung aus den Abbildungen (\ref{fig:detekscan}), (\ref{fig:zscan}) und (\ref{fig:rockscan}) entsprechen den erwarteten
Plots aus Abbildung (\ref{fig:beispiele}), daher lässt sich sagen, dass die Justage
erfolgreich durchgeführt werden konnte, allerdings wären die Ergebnisse vermutlich etwas genauer wären bei Detektor- und Z-Scan mehr Datenpunkte vorhanden. Dies wirkt sich beispielsweise auf den Geometriewinkel aus, denn die Strahlbreite, aus der der Geometriewinkel berechnet wird, kann nur relativ ungenau
aus Abbildung (\ref{fig:zscan}) ermittelt werden. Der gemessene Geometriewinkel $\alpha_\mathrm{g,Messung}$ weicht vom berechneten Geometriewinkel $\alpha_\mathrm{g,Theorie}$ ab mit:
\begin{align*}
  \alpha_\mathrm{g,Messung}&= \SI{0.5}{\degree} \\
  \alpha_\mathrm{g,Theorie}&= \SI{0.573}{\degree}
\end{align*}
Daraus ergibt sich eine Abweichung von ca. 12.7\%. \\

\noindent Der in Abbildung (\ref{fig:messungr}) zu sehende korrigierte Reflektivitätsscan liegt zwar in der selben Größenordnung, aber dennoch
über das ganze betrachtete Intervall unter dem Wert der Fresnelreflektivität von Silizium. Die erwarteten Kiessing-Oszillationen sind gut zu erkennen und auch die Näherung an
die Kurve mit Hilfe des Parrat-Algorithmus ist für $\alpha_\mathrm{i}>\SI{0.5}{\degree}$ gut gelungen. Allerdings sind die Parameter dafür manuell angepasst worden. Im Folgenden
sind die eingestellten Parameter mit den Literaturwerten \cite[5]{Anleitung3} verglichen. Zunächst wird die Dispersion sowie die Schichtdicke betrachet:
\begin{align*}
  \delta_\mathrm{PS,Parrat}&= \SI{0.55 e-6}{} \\
  \delta_\mathrm{PS,Literatur}&= \SI{3.5 e-6}{} \\
  \delta_\mathrm{Si,Parrat}&= \SI{6.6 e-6}{} \\
  \delta_\mathrm{Si,Literatur}&= \SI{7.6 e-6}{} \\
  d_\mathrm{PS,Messung}&=\SI{8.7 e-8}{\meter} \\
  d_\mathrm{PS,Parrat}&=\SI{8.6 e-8}{\meter}
\end{align*}
Während für Silizium die Dispersion relativ gut übereinstimmt mit einer Abweichung von ca. 13\%, ist die Abweichung beim Polystyrol deutlich höher mit ca. 84\%.
Die Schichtdicken, die einerseits aus dem Abstand der Minima der Kiessig-Oszillation und andererseits mit Hilfe des Parrat-Algorithmus berechnet wurden, stimmen fast exakt überein und sollten
daher eine gute Einschätzung für die tatsächlichen Dicke der Probe geben können. \\

\noindent Jetzt werden noch die kritischen Winkel betrachtet:
\begin{align*}
  \alpha_\mathrm{c,PS,Messung}&=\SI{0.060}{\degree} \\
  \alpha_\mathrm{c,PS,Literatur}&=\SI{0.153}{\degree} \\
  \alpha_\mathrm{c,Si,Messung}&=\SI{0.208}{\degree} \\
  \alpha_\mathrm{c,Si,Literatur}&=\SI{0.223}{\degree} \\
\end{align*}
Da die Winkel direkt aus den Dispersionen bestimmt wurden mit Gleichung (\ref{eqn:Totalreflexion}), hat auch hier wieder der Wert für Silizium die wesentlich geringere Abweichung.
In Abbildung (\ref{fig:messungr}) ist zu erkennen, dass die kritischen Winkel sehr gut mit den Intensitätseinbrüchen der jeweiligen Kurven übereinander liegen. \\

\noindent Die letzten beiden Parameter, die für den Parrat-Algorithmus verwendet wurden sind die Rauigkeiten, diese lauten:
\begin{align*}
  \sigma_\mathrm{1} &=\SI{7.5 e-10}{\meter} \\
  \sigma_\mathrm{2} &=\SI{6.3 e-10}{\meter} \\
\end{align*}
Dabei ist $\sigma_\mathrm{1}$ die Rauigkeit zwischen Luft und Polystyrol und $\sigma_\mathrm{2}$ die Rauigkeit zwischen Polystyrol und Silizium. Für diese Parameter liegen
keine Literaturwerte vor. Allerdings hat $\sigma_\mathrm{2}$ einen großen Einfluss auf den Verlauf der Kurve des Parrat-Algorithmus, daher ist anzunehmen, dass die
Rauigkeit $\sigma_\mathrm{2}$ relativ genau bestimmt wurde.\\

\noindent Insgesamt lässt sich sagen, dass über den Parrat-Algorithmus als auch über die Abstände der Minima ein gutes Ergebnis für die Schichtdicke erzielt wurde. Allerdings
war die Anpassung des Parrat-Algorithmus sehr subjektiv und es ist möglich, dass mit anderen Parametern eine bessere Annäherung an die Messung erzielt werden kann.
Weitere Gründe für Abweichungen können eine nicht optimale Justage der Apparatur oder nicht genügend Datenpunkte wie oben beschrieben.
