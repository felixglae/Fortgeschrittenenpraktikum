
\section{Theorie}
\label{sec:Theorie}
\subsection{Brechungsindex und Reflexion}
\label{sec:Brechungsindex}
Beim Auftreffen von Röntgenstrahlung vom Vakuum auf ein Medium, dessen Brechungsindex von eins abweicht, ist gegeben durch:
\begin{equation}
  \label{eqn:Brechungsindex}
  n=1-\delta+ i\cdot \beta
\end{equation}
\beta ist dabei die Absorption und \delta ist ein Korektur. Bei Röntgenstrahlung ist bei diesem Übergang eine Totalreflexion möglich, da der Brechungsindex in Materie kleiner als eins ist. Im allgemeinen Fall trifft die elektromagnetische Welle in einem Winkel $a_\mathrm{e}$ auf eine unendlich dicke, gomogenen, glatte Materialschicht.Ein Teil wird dann im Winkel $a_\mathrm{e}=a_\mathrm{r}$ reflektiert. Der restliche Teil wird unter dem Winkel $a_\mathrm{t}$ gebrochen und transmitiert. Die Totalreflexion tritt hierbei unter dem kritischen Winkel $a_\mathrm{c}$ auf. Dies bedeutet, dass keine Transmission auftritt. Dieser Kritische Winkel ist gegeben durch:
\label{sec:Brechungsindex}
Beim Auftreffen von Röntgenstrahlung vom Vakuum auf ein Medium, dessen Brechungsindex von eins abweicht, ist gegeben durch:
\begin{align}
  \label{eqn:Totalreflexion}
  a_\mathrm{c} &= \sqrt{2 \sigma} \\
  a_\mathrm{c} &= \lambda\cdot\sqrt{\dfrac{r_\mathrm{e}\cdot\rho}{\pi}}
\end{align}
