\section{Diskussion}
\label{sec:Diskussion}
In Abbildung \ref{abb:delta} ist der normierte Rotationswinkel $\vartheta_\mathrm{norm}$ gegen das Quadrat der Wellenlänge $\lambda^2$. Der Theorie nach sollte sich zwischen
diesen beiden Größen ein linearer Zusammenhang zeigen, dies bestätigt Abbildung \ref{abb:delta}. Allerdings weichen die einzelnen Messwerte oft von der Ausgleichsgerade ab, dennoch ist damit die Vermutung aus der Theorie im Experiment bestätigt worden.
Das Prinzip der klassischen Betrachtung mit Hilfe der Einführung der effektiven Masse aus Abschnitt \ref{sec:effektive_masse} ist ergo möglich und
eignet sich dafür die Leitungselektronen in Halbleitern zu beschreiben. \\
Für die effektive Masse $m^*$ im Verhältnis zu Elektronenmasse $m_\mathrm{e}$ ergab sich für die verschieden dotierten Proben der jeweilige Wert zu:
\begin{align*}
  \dfrac{m^*_{\mathrm{leicht}}}{m_\mathrm{e}} &= 0.074668 \pm 0.000005 \\
  \dfrac{m^*_{\mathrm{hoch}}}{m_\mathrm{e}} &= 0.086827 \pm 0.000006
\end{align*}
Im Vergleich dazu ist der Literaturwert \cite[1]{Bild6} gegeben durch:
\begin{equation*}
  \dfrac{m^*_{\mathrm{Lit}}}{m_\mathrm{e}}=0.067
\end{equation*}
Daraus ergibt sich eine Abweichung von etwa 11.4\% für die leichtdotierte Probe und bei der hochdotierten Probe eine Abweichung von etwa 29.6\%.
Daraus folgt, dass die Messung mit der leichtdotierten Probe die effektive Masse genauer bestimmen konnte, als die mit der hochdotierten Probe. Zu der relativ großen Abweichung kommt es
vermutlich durch das fehleranfällige Messverfahren. Unter Anderem war es schwer möglich ein exakt genaues Spannungsminimum am Oszilloskop zu finden, sodass die Winkel nicht immer genau bestimmt
werden konnten und eher aus einem Bereich von einigen Grad bzw. Minuten stammen. Eine weitere Fehlerquelle könnte das nicht ganz homogene Magnetfeld sein, denn dieses wird durch die
strominduzierte Wärme in den Spulen beeinflusst. Desweiteren gibt es trotz des Selektivverstärkers Rauscheffekte, wenn diese in der Größenordnung der jeweils betrachteten Frequenzen liegen.\\
Zusammenfassend lässt sich sagen, dass trotz der etwas fehleranfälligen Apparatur der verwendete Versuchsaufbau dafür geeignet ist die effekive Masse von Leitungselektronen zu bestimmen.
